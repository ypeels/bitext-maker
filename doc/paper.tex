% Yes, this requires flipping flags to compile for different conferences...
% But you can press Ctrl+T while editing in TeXworks without having to change windows!

% conditional compilation
\def\buildemnlp{0}
\def\buildamta{1}
\def\buildhcomp{2}
\def\buildtarget{\buildemnlp} % required to be a single character for \if

% emnlp2016.sty, included by header-emnlp.tex, does NOT tolerate ANY surrounding \else statements...
% an if/else if/... or ifcase statement would have been preferred...
% but \documentclass is declared in the header file, which results in some hidden error-checking:
    % unsupported \buildtarget will fail to build (``\usepackage before \documentclass'')
    % duplicate target values (e.g., emnlp=amta=0) will fail to build (``two \documentclass commands'')
\if \buildtarget \buildemnlp    
    % preamble from emnlp2016.tex to allow reuse of paper.tex between conferences

\documentclass[11pt,letterpaper]{article}
\usepackage{emnlp2016}
\usepackage{times}
\usepackage{latexsym}

% Uncomment this line for the final submission:
%\emnlpfinalcopy

%  Enter the EMNLP Paper ID here:
\def\emnlppaperid{137}

% To expand the titlebox for more authors, uncomment
% below and set accordingly.
% \addtolength\titlebox{.5in}    

\newcommand\BibTeX{B{\sc ib}\TeX}

%\title{Return of the Linguist:\\Toward scalable parallel corpus generation for machine translation}
%% according to my notes, I came up with this title around 3/11/16

% Author information can be set in various styles:
% For several authors from the same institution:
% \author{Author 1 \and ... \and Author n \\
%         Address line \\ ... \\ Address line}
% if the names do not fit well on one line use
%         Author 1 \\ {\bf Author 2} \\ ... \\ {\bf Author n} \\
% For authors from different institutions:
% \author{Author 1 \\ Address line \\  ... \\ Address line
%         \And  ... \And
%         Author n \\ Address line \\ ... \\ Address line}
% To start a seperate ``row'' of authors use \AND, as in
% \author{Author 1 \\ Address line \\  ... \\ Address line
%         \AND
%         Author 2 \\ Address line \\ ... \\ Address line \And
%         Author 3 \\ Address line \\ ... \\ Address line}
% If the title and author information does not fit in the area allocated,
% place \setlength\titlebox{<new height>} right after
% at the top, where <new height> can be something larger than 2.25in
%\author{Siddharth Patwardhan \and Daniele Pighin\\
%  {\tt publication@emnlp2016.net}}

\date{}

%%%\begin{document}

%%%\maketitle


    \newcommand\mycitep[1]{\cite{#1}}     % conference-specific ``glue code''
    \newcommand\mynewcite[1]{\newcite{#1}}
    \newcommand\amtaonly[1]{}
    \newcommand\hcomponly[1]{}
\fi
\if \buildtarget \buildamta    
    % preamble from amta2016.tex to allow reuse of paper.tex between conferences

\documentclass[]{article}
\usepackage[letterpaper]{geometry}
\usepackage{amta2016}
\usepackage{times}
\usepackage{url}
\usepackage{latexsym}
\usepackage{natbib}
\usepackage{layout}

\newcommand{\confname}{AMTA 2016}
\newcommand{\website}{\protect\url{http://www.amtaweb.org/}}
\newcommand{\contactname}{research track co-chair Lane Schwartz}
\newcommand{\contactemail}{lanes@illinois.edu} 
\newcommand{\conffilename}{amta2016}
\newcommand{\downloadsite}{\protect\url{http://www.amtaweb.org/}}
\newcommand{\paperlength}{$12$ (twelve)}
\newcommand{\shortpaperlength}{$6$ (six)}

%% do not add any other page- or text-size instruction here

\parskip=0.00in

\begin{document}

% \mtsummitHeader{x}{x}{xxx-xxx}{2016}{45-character paper description goes here}{Author(s) initials and last name go here}
\title{\bf Formatting Your Paper \\
  for the \confname~Conference}  

\author{\name{\bf First Author} \hfill  \addr{author1@abc.university.country}\\ 
        \name{\bf Second Author} \hfill \addr{author2@abc.university.country}\\ 
        \addr{Department of Science, My University, 
        MyTown, Zip, Country}
\AND
       \name{\bf Third Author} \hfill \addr{author3@abc.university.country}\\
        \addr{Department of Science, My University, 
        MyTown, Zip, Country}
}

\maketitle
\pagestyle{empty}


    \newcommand\mycitep[1]{\citep{#1}} 
    \newcommand\mynewcite[1]{\cite{#1}}  
    \newcommand\amtaonly[1]{#1}
    \newcommand\hcomponly[1]{}
\fi
\if \buildtarget \buildhcomp    
    \def\year{2016}\relax
%File: formatting-instruction.tex
\documentclass[letterpaper]{article}
\usepackage{aaai16}
\usepackage{times}
\usepackage{helvet}
\usepackage{courier}
\frenchspacing
\setlength{\pdfpagewidth}{8.5in}
\setlength{\pdfpageheight}{11in}
\pdfinfo{
/Title (Insert Your Title Here)
/Author (Put All Your Authors Here, Separated by Commas)}
\setcounter{secnumdepth}{2}  % 0-no numbers, 1-section numbers, 2-section and subsection numbers
% \begin{document}
% The file aaai.sty is the style file for AAAI Press 
% proceedings, working notes, and technical reports.
%
\title{Formatting Instructions \\for Authors Using \LaTeX{}}
\author{AAAI Press\\
Association for the Advancement of Artificial Intelligence\\
2275 East Bayshore Road, Suite 160\\
Palo Alto, California 94303\\
}
    \newcommand\mycitep[1]{\cite{#1}} 
    \newcommand\mynewcite[1]{\citeauthor{#1}~\shortcite{#1}}  
    \newcommand\amtaonly[1]{}
    \newcommand\hcomponly[1]{#1}
\fi
    
% Add support for snippets of \zh{中文}
\usepackage{CJK}
\usepackage[utf8]{inputenc}
\newcommand\zh[1]{\begin{CJK}{UTF8}{gbsn}#1\end{CJK}}

% this would allow clickable urls for my convenience - but many conferences forbid them?
% \usepackage{hyperref}


\begin{document}
\maketitle




\begin{abstract}
Hello, World!

Here is some normal spacing.

\zh{你好,世界!}

Scalable parallel corpus generation
\end{abstract}




Curiosity: how many AMTA pages are the EMNLP instructions? i.e., how long would my 4-page EMNLP paper be?


\section{Introduction}

RNN encoder-decoders are not the first corpus-based MT methodology (EBMT then SMT then this), and it is unlikely in this era of Big Data that they will be the last.
Focusing on corpus generation allows linguists to focus on capturing the most statistically prevalent linguistic phenomena, while insulating them from the strengths and weaknesses of various implementations.
For instance, EBMT suffers from relatively poor generalization and SMT has problems with nonlocality.
(RNN built on relatively shaky ground, since word vectors don't QUITE make sense for function words?)
 
Such synthetic parallel corpora can also be viewed as benchmark tasks (need to cite Facebook bAbI).

One might conjecture that the fundamental problem in natural language processing is one of data representation


\section{Related Work}
\label{sec:related}

Nice tip: ONE LINE PER SENTENCE.

Unlike more AI-oriented or natural language understanding approaches, we are solely concerned here with the generation of surface forms

At some point, I will almost definitely be citing Moses \mycitep{moses}

the latest version of the Stanford Segmenter \mycitep{segmenter:2005,segmenter:2008}

stanford POS tagger (incomplete bibtex entry?) \mycitep{postagger:2003}

This aspect of our work is perhaps closest in spirit to the equivalence classes used by \mynewcite{brown:99} to reduce the data requirements of an EBMT system \mycitep{ebmtbook}.

In the Moses toolchain, the {\small \tt mkcls} component \mycitep{och:98} performs this process in reverse, binning words into automatically-generated equivalence classes to alleviate data sparsity issues.
% USUALLY alignment only, CAN be applied to translation... http://www.statmt.org/moses/?n=Advanced.Models#ntoc6
With some effort, you can use these bins during steps other than alignment, although this is not done by default.
...
Moreover, as they are upstream of GIZA++, these bins are constructed at the lexical level, not the phrase level, whereas the templates described below can make substitutions at the level of larger constituents and phrases.


some neural network MT paper?

aha, yoshua bengio's "curse of dimensionality" - becomes a blessing?  
Unlike many machine learning tasks (image recognition, finance), NLP data are purely artificial/manmade.  New paradigm?  Not enough attention has been paid IMO to semi-automated corpus generation.  Well, typical approaches are paid, like Mechanical Turk... but there is a vast untapped source of data in the form of student problem sets, in the spirit of reCAPTCHA (citation!?)

Translation is perhaps the only NLP task for which millions of people already undergo years of training

other approaches to synthetic corpora, which are essentially bootstrapped from other MT systems

Hardly a new idea ... but the emphasis here is on the resulting corpus




maybe the latest Hutchins overview or nutshell?

fire a linguist and accuracy goes up

collobert and weston: learn words from letters. interesting from an AI perspective, seems unlikely to improve performance in the near future

chomsky: transformations, colorless green ideas


size doesn't always matter - those massive OpenSubtitles corpora don't help
- uh, is the problem quality? should i run some Moses runs with a small portion of MultiUN to see how quality improves?

lots of things are resolved naturally when considering generation instead of discrimination
- ambiguity (you know what you want to say)
- segmentation (you know what ``phrases'' you wanted to output)


Can think of this as something intermediate between fully automatic (RBMT, or semi-supervised SMT) and fully hand-crafted (human translation, which is the norm).
Compositional trees are more automatic-like, but generate more sentence pairs/tuples.
Complex templates are more hand-crafted, but generate fewer pairs/tuples, and are (presumably) harder to combine compositionally (but are they really!?).

The tree structure we adopt is something intermediate between constituency trees and dependency trees, and it was chosen for ease of implementation.


\section{Model}

\subsection{Simplistic Linguistics}
We adopt the exceedingly simplistic view of language as nouns, verbs, and descriptions (modifications) thereof.

two sources of complexity
- lexical (semantic?)
- syntactic 

The fact that verbs can serve as nouns (gerundives) and modifiers (participles and modals) is a significant source of added complexity. (footnote?)


I guess I should use placeholder graphics for now, since deadlines for camera-ready are quite a bit later.
EMNLP: 6/3 submission, 7/29 notification, 9/23 camera-ready
AMTA: 6/27 submission, 8/8 notification, 9/12 camera-ready

Testing section reference: \ref{sec:related} - this requires changing secnumdepth=2 in AAAI

\section{Experimental Methods}

citations needed
IWSLT - point out that this corpus is freely available
OPUS zh\_zh
MultiUN
Moses

parameters used

IWSLT: TED talks are a series of lectures, which makes them something intermediate between the conversational language of movie subtitles and the formal written language of UN resolutions.
Should I mention that you can kind of discern this from their POS distributions? I would probably have to mention something concrete then... But yeah, I guess it's worth mentioning if I have space, which I almost certainly do for AMTA

heuristically purged zh\_zh, which is a very noisy corpus

\section{Results}

2016-02-20 learning curves adding only 130,000 in-domain lines improves scores by like 2.0 BLEU points

2016-03-03-225556 1M zh\_zh lines made scores slightly WORSE?

2016-04-29-093202 1M MultiUN lines only improved scores SLIGHTLY (by like 0.4 BLEU points)

Quantity and maybe even quality (you'd expect MultiUN translations to be better?) matters much less than having data in-domain

Of course, I need to redo (ALL?) these jobs with final segmenter choice

Evaluated using the IWSLT 2015 progress data set (tst2014) to allow comparison with both 2015 and 2014 systems.


\section{Future Work}

I.e., shit I REALLY doubt I have time for...

For simplicity, the current implementation is lexically one-to-one; for scalability in the number of languages, you ideally want many-to-many.

For this paper, the lexical database was drawn from the IWSLT training set to maximize bottom-line performance.
[i.e., i don't have TIME to do a general-purpose database]
This is indeed one potential use of this system, for training set amplification.
The envisioned ultimate target domain is (cleaned) everyday speech (maybe movie or TV subtitles?) , for which syntactic and lexical diversity is expected to be relatively low, and unknown words would be particularly suitable for crowdsourcing.

Localization of semantic taxonomy - this would have to be one-to-one, and you'd think it would clutter up the raw text UI...


Instead of the next paragraph, should probably just point out how rudimentary the current ``grammar'' is. 
Future development may be facilitated by adapting existing work from RBMT or NLG. 
Okay, I guess just point out this grammar doesn't even handle something as basic as:

Shit tons of linguistic phenomena that aren't even that uncommon, including but not limited to:
More thorough treatment of articles.
(More) treatment of prepositional phrases.
Conjunctions.
Interrogatives and imperatives.
Pronouns - implemented as names here...? wait, but what about I/me? ugh



\amtaonly{
\section{AMTA}
This text only gets added for AMTA. I think it'd be nice to have this for ``filler'' text

If I really am submitting to this conference too, then I can afford (and probably want to?) draw out the length, 
} % end \amtaonly



\hcomponly{ % hmm, maybe I should use this stuff to pad AMTA too? well, let's not pad just for padding's sake... spare the reviewers
\section{HCOMP}

An open challenge to the community - making a crowdsourcing platform
Should I do a quick back-of-the-envelope calculation on how many child-hours are spent on language-related problem sets?

We believe that for crowdsourcing to be most fruitful, there should a matching incentive to the crowd... 
At the risk of sounding like a sales pitch, what if the crowd wanted to pay you to give you data, like Tom Sawyer?
The 2015 paper only tagged like 200 words... well, they were trying something WordNet style... but it just goes to show how slow it is...
WordNet has shit tons of senses, but only a handful of them are common... we want the most common senses here, not completeness.
For rare senses, we can make custom templates.
The semantics we desire are more lightweight (?) - just enough to generate surface forms; don't need to do complex logical reasoning on it.

Rare nouns are the biggest source of lexical diversity.
VERY expensive to do with a small team.
Very ripe for crowdsourcing, but not necessarily by MOOCs, as proposed by LearnWork - problem sets seem more appropriate here.
With a sufficiently compelling problem set app, schools could possibly be willing to pay subscription fees to use this app... (hmm, I see a problem - schools don't have that much money... well, that's where scale comes into play?) well, at the very least, if you make it free, you still get their data
The question seems not so much ``should we'' as much as ``why aren't we doing this yet?'' - at least in the context of NLP.
Unlike some traditional machine learning tasks, which use either sensory or observed data as input, human writers ARE the primary and most natural source of linguistic data.
You wouldn't want to crowdsource bitmaps of cats.
And even if you were crowdsourcing photos of cats, or even photo labeling, that's not really something students do beyond preschool.
But to master one's native tongue takes like 10 years, and there's a BUNCH of unharnessed labor there.

} % end \hcomponly





% these are so short that it's probably better to have it in literally instead of using \input
\if \buildtarget \buildamta    
    \small
    \bibliographystyle{apalike}
\else \if \buildtarget \buildemnlp
    %Note that emnlp2016.bst chokes if there are 0 citations in your paper.    
    %\mynewcite{moses}    
    \bibliographystyle{emnlp2016}
    
\else \if \buildtarget \buildhcomp    
    \bibliographystyle{aaai}
\else
    % some extra error checking, which could not be used with emnlp2016.sty
    \GenericError{}{Invalid buildtarget - \buildtarget}
\fi \fi \fi

\bibliography{references}

\end{document}
